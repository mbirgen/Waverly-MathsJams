\section*{Recreational Mathematics}
\begin{enumerate}
	\item There is a unique real number \(x\) that can be expressed in the following form:
	\[ x= 1+\frac{1}{1+\frac{1}{\frac{1}{1+ \dots}}},\]
	where the dots ``\(\dots\)'' mean ``and so on, forever.'' What is this number \(x\)? \vfill
	
	\item Take a strip of paper and fold it in half several times as described. 
	\begin{quote}
	Imagine taking the ends in your hands and placing the right hand end on top of the left. Now press the strip flat so that it is
folded in half and has a crease. Repeat the whole operation on the new strip. . . .

After folding the strip in half a number of times, the strip should be unfolded by exactly undoing the folding process (this is important to note, because different unfolding methods can result in different sequences of creases).
	\end{quote}
	
\begin{enumerate}
	\item Afrer 1 fold, the paper will have 1 crease. After 2 folds, the paper will have 3 creases. 
	How many creases will there be after \(n\) folds? \vfill
	\item Unfold it and observe that some of the creases are IN and some are OUT. For example, three folds produce the sequence 
	\begin{center}
	in in out in in out out
\end{center}
What sequence would arise from 10 folds (if that many were possible)?
\vfill
\end{enumerate}
\end{enumerate}