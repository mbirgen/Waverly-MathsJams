\begin{center}
	 \textbf{Complements of Kappa Mu Epsilon\\ (Math Club)}
\end{center} \normalsize
\textbf{Instructions:} The goal is to fill in the numbers \textbf{1} through \textbf{81} in the grid so that the numbers increase in the snake where the snake can only go up, down, left, or right.  For example, the blank box in the upper left corner must be a 51 because 51 has to be next to 50 and the 50 is stuck with only one blank box next adjacent.

\begin{center}
	\renewcommand{\arraystretch}{1.5} %\large
	\begin{tabular}{*{9}{|c}|} \hline
	49 & 48 & 47 & 46 & 45 & 4 & 5 & 6 & 7\\ \hline 
50 &  &  &  &  &  &  &  & 8\\ \cline{2-2}\hline 
81 &  &  &  &  &  &  &  & 13\\ \hline 
76 &  &  &  &  &  &  &  & 14\\ \hline 
75 &  &  &  &  &  &  &  & 19\\ \hline 
66 &  &  &  &  &  &  &  & 20\\ \hline 
65 &  &  &  &  &  &  &  & 21\\ \hline 
64 &  &  &  &  &  &  &  & 28\\ \hline 
63 & 62 & 61 & 60 & 33 & 32 & 31 & 30 & 29 \\ \hline 
	\end{tabular}
\end{center}
Like Math, Computer Science, Physics on Facebook : \texttt{https://tinyurl.com/MCSPFacebook}